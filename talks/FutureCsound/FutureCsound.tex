\documentclass{beamer}

\mode<presentation>
{
  \usetheme{Warsaw}
  \definecolor{links}{HTML}{2A1B81}
  \hypersetup{colorlinks,linkcolor=,urlcolor=links}
  %\usetheme{Frankfurt}
  %\usecolortheme{seagull}
  % or ...
  \setbeamercovered{transparent}
  % or whatever (possibly just delete it)
}

\usepackage[english]{babel}
\usepackage[utf8]{inputenc}
\usepackage{times}
\usepackage[T1]{fontenc}
\usepackage{fancyvrb}
\usepackage{listings}
\usepackage{graphicx}
\usepackage{attachfile}

\title[Future Csound] % (optional, use only with long paper titles)
{The Future of Csound}

\author[Gogins] % (optional, use only with lots of authors)
{Michael Gogins\\
\url{http://michaelgogins.tumblr.com} 
}
% - Give the names in the same order as the appear in the paper.
% - Use the \inst{?} command only if the authors have different
%   affiliation.

\institute[Irreducible Productions] % (optional, but mostly needed)
{
  Irreducible Productions\\
  New York
}
% - Use the \inst command only if there are several affiliations.
% - Keep it simple, no one is interested in your street address.

\date[6 September 2017] % (optional, should be abbreviation of conference name)
{6 September 2017}
% - Either use conference name or its abbreviation.
% - Not really informative to the audience, more for people (including
%   yourself) who are reading the slides online

\subject{Csound}
% This is only inserted into the PDF information catalog. Can be left
% out. 
\begin{document}
\lstset{basicstyle=\ttfamily\tiny,commentstyle=\ttfamily\tiny,tabsize=2,breaklines,fontadjust=true,keepspaces=false,fancyvrb=true,showstringspaces=false,moredelim=[is][\textbf]{\\emph\{}{\}}}

\begin{frame}
  \titlepage
\end{frame}

\begin{frame}{What Csound Is}
    \begin{itemize}
        \item     
            To summarize for clarity: \textit{Csound is a sound programming 
language and sound processing system}.
        \item
            Csound is used mainly for making computer music, but has other 
uses such as data sonification, audio in computer games, 
education, \textit{etc.}, and could have new uses in future.
        \item
            All functionality of Csound is available in the Csound shared 
libary, with interfaces to most widely used programming languages.
        \item
            Csound runs on most widely used operating systems, and on the Web.
        \item
            Csound is old, going back to 1985, and has a long tradition with a 
large body of work that still works thanks to backwards compatibility.
        \item 
            Csound is open source.
    \end{itemize}    
\end{frame}

\begin{frame}{What Csound Could Be}
    \begin{enumerate}
        \item \textit{Software fragments}. Csound could unify the world of 
music programming by serving as a standard sound engine. Not likely but the 
closer the better.
        \item \textit{Music fragments}. Csound could serve as a \textit{lingua 
franca} for computer music. Not likely but the closer the 
better.
        \item Other systems vie for this role, notably:
            \begin{description}
             \item[Max] Widely taught because widely used for 
commercial sound design.
             \item[SuperCollider] Has a younger hipper community.
             \item[WebAudio] Unit generator based audio synthesis in standard 
HTML5. Probably will become more important.
            \end{description}
        \item But Csound is the most powerful, and has advantages that will 
become more important in the future, such as multi-threading, and an existing 
WebAssembly build.
    \end{enumerate}    
\end{frame}

\begin{frame}[allowframebreaks]
    \frametitle<presentation>{Problems}
    \begin{enumerate}
        \item     
            The Csound language is still clunky and lacks useful features such 
as structures, classes, and closures.
        \item
            The Csound infrastructure is not complete on all platforms.
        \item
            The Csound API is not consistent on all platforms, especially 
HTML5.
        \item
            The Csound build has a lot of external dependencies.
        \item
            We need more developers who understand the system.
    \end{enumerate}    
\end{frame}

\begin{frame}[allowframebreaks]
    \frametitle<presentation>{Solutions}
    \begin{enumerate}
        \item     
            Add structures, classes, and closures to the Csound language.
        \item
            Build all releases using online continuous integration systems.
        \item
            That means using a package manager or automated builds for all 
third party dependencies.
        \item
            When that's working, simplify the build system.
        \item
            When that works, write scripts to do local builds by emulating the 
CI build.
    \end{enumerate}    
\end{frame}

\begin{frame}[allowframebreaks]
    \frametitle<presentation>{Further Out}
    \begin{enumerate}
        \item     
            Try to get Csound to become an international standard. SAOL and 
SASL were a stab in this direction that failed due to lack of a performant 
reference implementation. Csound is performant enough. Not \textit{all} of 
Csound needs to be in such a standard.
        \item
            Consider exposing Csound opcodes as WebAudio unit generators.
        \item 
            Make a WebAssembly-based showcase for Csound examples with a 
real ``Wow!'' factor. Google's 
\href{https://experiments.withgoogle.com/chrome}{Chrome Experiments} and the 
\href{https://www.shadertoy.com/}{Shadertoy} website are models.
    \end{enumerate}    
\end{frame}

\begin{frame}[allowframebreaks]
    \frametitle<presentation>{Recruiting Developers}
    \begin{enumerate}
        \item     
            Follow through on the CI-oriented build systems for OS X, Linux, 
and Windows. Cut down the overhead of getting Csound building!
        \item   
            Maynooth might be a template, consider working with teachers in 
computer music courses who for various reasons might be amenable to typing 
code instead of eating Max spaghetti.
        \item   
            Could this work together with Google Summer of Code?
        \item   
            Could people volunteer to help actual composers who might be 
interested in using Csound but are scared of the learning curve?
    \end{enumerate}    
\end{frame}

\begin{frame}[allowframebreaks]
    \frametitle<presentation>{For Me Personally}
    \begin{enumerate}
        \item     
            For me personally, what I need from Csound is as 
always...
        \item   
            Good stuff I can beg, borrow, or steal.
        \item   
            Fast working speed. I want Csound to be my playpen.
        \item   
            Not so much for sound design as for live performance, I've been 
inspired by watching people patch modular analog synthesizers on stage. 
Something like that would be nice for Csound and then good patches could be 
saved and used in through-composed pieces.
    \end{enumerate}    
\end{frame}

\end{document}


